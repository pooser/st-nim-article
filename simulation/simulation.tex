\section{GEANT4 Simulations} \label{sec:sim}

In this section, the Monte Carlo (MC) simulations that were conducted in order to better understand the performance and characteristics of scintillators machined to the nominal \gx{} Start Counter design are discussed.  Comparisons were made with the data observed in experiments conducted on the bench.  The MC studies were performed with the use of the simulation tool-kit GEANT4 which simulates the passage of particles through and interacting with matter \cite{geant4_website}.

\subsection{Simulating a Simplified Model of the ST} \label{sec:sim_simple}

As discussed in Sec.~\ref{sec:design_paddles}, the ST scintillator paddles have a unique geometry in which the nose section tapers in width as the paddles approach the beam line.  This tapering effect results in a unique phenomenon in which the light output of the scintillator paddle begins to increase as the source moves further away from the readout detector.  At first, this phenomenon is completely contrary to what one might expect. When the source moves further away from the end being readout, the photons have a larger effective path length and thus, have an increased probability in being lost for detection.  However, this is antithetical relative to what is observed.

A primitive GEANT4 simulation was conducted to investigate the aforementioned phenomenon. For simplicity, only the two trapezoidal regions of a machined scintillator paddle were considered.  Namely, the wide straight section and the tapered nose section which are illustrated in the GEANT4 event display  seen in Fig. \ref{fig:beam_off}.
	\begin{figure}[!htb]
	\centering
	\includegraphics[width=1.0\columnwidth]{simulation/figs/beam_off}
	\caption{Simulated straight \& nose section geometries.  Shown is the GEANT4 event display.  Left: wide straight section.  Right: tapered nose section.  The sections have been oriented such that they are in the same coordinate system as defined in HallD.  The yellow lines are the scintillator boundaries, while the red lines are the boundaries of the SiPM.}
	\label{fig:beam_off}
	\end{figure}  

The EJ-200 scintillator material ($\rho=1.023~\mathrm{g/cc^{3}}$, $n = 1.58$) \cite{ej200_specs} was simulated with only one free parameter utilized to characterize the scintillator bar \textit{i.e.}, the reflectivity of the \textit{G4LogicalSkinSurface}, was set to 98\% so there remained some finite probability that photons could be lost in the scintillator medium.  Furthermore, the SiPM readout detector was placed at the upstream end of the two sections, seen in Fig. \ref{fig:beam_off}.  Moreover, the SiPM was constructed as a \textit{G4SensitiveDetector} made of Silicon with a 100\% detection efficiency.  The SiPM was constructed to have an active area of $\mathrm{3 \times 12~mm^2}$ which is identical to the readout system described in Sec.~\ref{sec:design_sipms}.

In order to simulate a charged particle traversing through the scintillator medium resulting in the production of photons along its path through the material, optical photons were generated inside the volume of the simulated scintillator material.  The scintillation yield was defined to be $10,000 \gamma 's / $ 1 MeV \cite{ej200_specs}. For visual purposes, Fig. \ref{fig:100_events} shows 100 optical photons being produced at the tip of the far end of the two sections of the simulated scintillator paddle.
	\begin{figure}[!htb]
	\centering
	\includegraphics[width=1.0\columnwidth]{simulation/figs/100_events}
	\caption{100 Optical photons generated in the staight \& nose sections.  Left: wide straight section.  Right: tapered nose section.  The neon green lines are the paths of the optical photons.  It is clear that some photons do in fact escape the scintillator medium, while others are collected in the simulated SiPM detector.}
	\label{fig:100_events}
	\end{figure}

In order to sample the entirety of the two sections, 10,000 optical photons were generated at 16 different locations inside the medium of the scintillator. 
	\begin{figure}[!htb]
	\centering
	\includegraphics[width=1.0\columnwidth]{simulation/figs/gun_locations}
	\caption{Optical photon gun locations along the straight \& nose sections. Left: wide straight section.  Right: tapered nose section.  The magenta geometries indicate the scintillator boundaries of the two sections.  The red box is the sensitive SiPM detector, and the green cylinders represent the location of the 16 optical photon gun locations. The locations of the source were chosen to be equal distances apart relative to each of the two sections.}
	\label{fig:gun_locations}
	\end{figure}
The photon energies ranged between $0.5 - 3.0$ eV \cite{krane_ch7} and were generated randomly in $4\pi$ along a 3 mm path $(y-axis)$ in the scintillator medium.  The path was oriented orthogonal to the wide surface of the scintillator.  In essence, this simulates a charged particle traversing through the medium with a $\theta_{track} = 90^{\circ}$ in hall coordinates.  The number of photons collected by the SiPM at each of the 16 source locations is counted and correlated to the source location.  The results can be seen in Fig. \ref{fig:sim_results}.
	\begin{figure}[!htb]
	\centering
	\includegraphics[width=1.0\columnwidth]{simulation/figs/sim_results}
	\caption{Simulation results for simplified two section scenario. The total number of photons which were collected by the SiPM detector at each of the 16 source locations is plotted against the sources distance from the sensitive detector. Left: wide straight section.  Right: tapered nose section.}
	\label{fig:sim_results}
	\end{figure}
From the data it is clear that the geometry of the nose section results in an improvement of light collection as the source moves further away from the readout detector.  In fact, there is a factor $\approx 1/2$ light loss observed in the straight section upon comparing the number of hits collected at the closest and furthest locations relative to the readout detector.  However, there is factor $\approx 3/2$ light gain observed in the nose region.  These results are primarily due to the unique scintillator geometry.  This phenomenon is advantageous in the case of the ST since the majority of forward going charged particles will traverse through the nose region.

\subsection{Simulating Machined Scintillator Geometry} \label{sec:sim_mach}

Further simulations were conducted to simulate more realistically the effects of light collection that results from the ST scintillator geometry and optical surface quality.  The ST scintillator geometry was imported into GEANT4 from a Vectorworks CAD drawing utilizing the CADMesh utility \cite{cadmesh_g4} and is shown \textit{via} the GEANT4 event display in Fig. \ref{fig:pk_cadmesh}.
	\begin{figure}[!htb]
	\centering
	\includegraphics[width=1.0\columnwidth]{simulation/figs/pk_cadmesh}
	\caption{Scintillator geometry imported into GEANT4 utilizing CADMesh Utility.  The scintillator is  coupled to a SiPM detector.  Left: isometric view.  Right: top view.  The tapering of the nose section is clearly visible.}
	\label{fig:pk_cadmesh}
	\end{figure}
The SiPM was constructed as a $12 \times 12 \times\ 10\ \mathrm{mm^{3}}$ volume with a 100 $\mathrm{\mu m}$ air gap between it and the wide end of the straight section.  Furthermore, the volume surrounding the scintillator volume was air.  The EJ-200 scintillator material, SiPM silicon detector, and optical photons were defined in an identical manner discussed in Sec.~\ref{sec:sim_simple}.

To simulate the imperfections of the scintillator surfaces due to manufacturing and machining, an optical surface ``skin'' was defined.  The ``skin'' material was defined to be of they type ``dielectric-dielectric'' and made use of the UNIFIED physics model \cite{scint_surface_sim} to define an imperfect scintillator surface.  Both the transmission efficiency and reflection parameters were implemented as free parameters to study their various effects on light transmission..  

The UNIFIED model allows one to define the finish of the scintillator surface as \textit{polished}, \textit{ground}, or \textit{unified} and is illustrated in Fig.~\ref{fig:polished_vs_ground} \cite{scint_surface_sim}.
	\begin{figure}[tbph]
	\centering
	\includegraphics[width=1.0\columnwidth]{simulation/figs/polished_vs_ground}
	\caption{UNIFIED Model of scintillator surfaces.  Left: Polar plot of the radiant intensity of the polished (left) and ground (right) models.  Right: Polar plot of the radiant intensity in the UNIFIED model.}
	\label{fig:polished_vs_ground}
	\end{figure}

In the polished model, Fresnel reflection and refraction is assumed, where as the ground model allows for Lambertian reflection, Fresnel refraction, backscattering, as well as spike and lobe reflections.  The spike ($C_{ss}$) reflection parameter assumes the optical photons are reflected as if the surface was a perfect mirror.  The backscattering ($C_{bs}$) reflection parameter assumes the photon is reflected in the same direction of incidence.  The Lambertian ($C_{dl}$) reflection parameter assumes that the photons are reflected corresponding to a Lambertian distribution.  The lobe ($C_{sl}$) reflection parameter assumes that the photons will reflect based on the orientation of the micro-facet on the scintillator surface, where $\sigma_{\alpha}$ defined the standard deviation of the distribution of the micro-facets orientation \cite{scint_surface_sim}.  One caveat of the aforementioned models is that they assume identical parameters for the entire optical surface \cite{puneet_sim_wiki}.

As was done in section \ref{sec:sim_simple} 10,000 optical photons were generated in the scintillator medium every 2.5 cm and the number of hits collected in the SiPM were recorded.  The results of these simulations are show in Fig. \ref{fig:transm_eff_vs_sig_alpha}.
	\begin{figure}[!htb]
	\centering
	\includegraphics[width=1.0\columnwidth]{simulation/figs/transm_eff_vs_sig_alpha}
	\caption{UNIFIED Model results.  Left: polished model while varying the transmission efficiency.  Right: ground model with varying $\sigma_{\alpha}$ which characterizes the standard deviation of the surfaces micro-facet orientation.}
	\label{fig:transm_eff_vs_sig_alpha}
	\end{figure}
It is clear that while assuming a polished surface and the transmission efficiency is increased, the amount of light collected in the SiPM also increases.  This is illustrated in Fig.~\ref{fig:transm_eff_vs_sig_alpha}.  Similarly, as the number of micro-facet orientations increase, meaning a more coarsely ground surface, the amount of light collection in the SiPM decreases as expected.  Moreover, in the instances where the surface quality of the machined scintillators are good, the phenomenon of light increase in the nose region as the source moves further from the readout detector is clearly observed.

