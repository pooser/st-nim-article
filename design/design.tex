\section{Design} \label{design}
\subsection{Overview}
The Start Counter (ST) detector, seen in figure \ref{fig:st-geo}, surrounds a 30 cm long super cooled liquid $\mathrm{H_{2}}$ target while providing $\sim 90 \%\ \mathrm{of}\ 4 \pi$ solid angle coverage relative to the target center.
\begin{figure}[!htb]
	\centering
	\includegraphics[width=1.0\columnwidth]{design/figs/30_Element_Pieces_Removed}
	\caption[Start Counter geometry]{Start Counter geometry.}
	\label{fig:st-geo}
\end{figure}
The primary purpose of the ST detector is, in coincidence with the tagger, to properly identify the electron beam bucket associated with detected particles produced \textit{via} linearly polarized photons incident on the target. It is designed to operate at tagged photon intensities of up to $10^{8}\,\gamma/s$ in the coherent peak.  Moreover, the ST has a high degree of segmentation for background rejection, is utilized in particle identification, and is a primary component of the level 1 trigger of the GlueX experiment.

EJ-200 scintillator material from Eljen Technology, which provides a decay time on the order of $2\ ns$ and a long attenuation length, was used to properly identify the beam buckets which are about $2\ ns$ apart\cite{pooser16}.  The ST detector consists of an array of 30 scintillators with pointed ends that bend towards the beam at the downstream end. The amount of support structure material was kept to an absolute minimum in the active region of the detector and is made up of low density Rohacell\textregistered. Silicon Photomultiplier (SiPM) detectors were selected as the readout system. The detectors are not affected by the high magnetic field produced by the superconducting solenoid magnet. Moreover, the SiPMs were placed as close as possible, \textit{i.e.} $< 250\ \mu m$, to the upstream end of each scintillator element, thereby minimizing the loss of scintillation light\cite{pooser16}.

\subsection{Scintillator Paddles}

Each individual paddle of the Start Counter was machined from a long, thin, polyvinyltoluene plastic EJ-200 scintillator bar that was diamond milled to be $600\ mm$ in length, $3\ mm$ thick, and $20 \pm 2\ mm$ wide, by Eljen Technology.  Each scintillator was bent around a highly polished aluminum drum by applying localized infrared heating to the bend region.  The bent scintillator bars were then sent to McNeal Enterprises Inc., a plastic fabrication company, where they were machined to the following desired geometry (see figure~\ref{fig:Scintillator Geometry}).
\begin{figure}[!htb]
	\centering
	\includegraphics[width=0.7\columnwidth,angle=270.]{design/figs/Scint_Geo}
	\caption[Start counter single paddle geometry]{Start counter single paddle geometry.}
	\label{fig:Scintillator Geometry}
\end{figure}

The paddles consist of three sections and is described from the upstream to the downstream end of the target.  The straight section is $394.65\ mm$ in length and is parallel to the target cell.  The bend region is a $18.5^{\circ}$ arc of radius $120\ cm$ and is downstream of the straight section. The tapered nose region is downstream of the target chamber and bends towards the beam line such that the tip of the nose is at a height of $2\ cm$ above the beam line.  

After the straight bar is bent to the desired geometry, the two flat surfaces that are oriented orthogonal to the wide, top and bottom, surfaces are cut at a $6^{\circ}$ angle.  During this process the width of the top and bottom surfaces are machined to be $16.92\ mm$ and $16.29\ mm$ wide respectively.  Each of the 30 paddles may be rotated $12^{\circ}$ with respect to the paddle that preceded it so that they form a cylindrical shape with a conical end.  The geometry of the ST increases solid angle coverage while minimizing multiple scattering.  

The 30 scintillator paddles are placed atop a Rohacell support structure which envelopes the target vacuum chamber seen in figure \ref{fig:ST_Cross_Section}.  
\begin{figure}[!htb]
	\centering
	\includegraphics[width=1.0\columnwidth]{design/figs/ST_Cross_Section}
	\caption[Cross section of the ST]{Cross section of the ST.}
	\label{fig:ST_Cross_Section}
\end{figure}
Rohacell is a rigid, low density foam ($\rho = 0.075\ g/cm^{3}$).  The Rohacell, which is $11\ mm$ thick, is rigidly attached to the upstream support chassis and extends down the length of paddles however, not to include the end of the conical section.  Glued to the inner diameter of the Rohacell support structure are 3 layers of carbon fiber ($\rho = 1.523\ g/cm^{3}$) which are $650 \mu m$ thick.  A cross section of the ST can be seen in figure \ref{fig:ST_Cross_Section} where the carbon fiber is visible.  The carbon fiber serves as to provide additional support during the assembly process as well as long term rigidity.  

The SiPM detectors are held in a fixed position while being attached to the lip of the upstream chassis \emph{via} two screws.  The scintillators are placed as close as possible to the active region of the SiPMs (see figure~\ref{fig:Partially Assembled Start Counter}).
\begin{figure}[!htb]
	\centering
	\includegraphics[width=1.0\columnwidth]{design/figs/Labeled_Partially_Assembled_Detector}
	\caption[Partially assembled Start Counter]{Partially assembled Start Counter.}
	\label{fig:Partially Assembled Start Counter}
\end{figure}

The ST scintillators are coupled \emph{via} an air gap ($< 250 \mu m$) into groups of four SiPMs set in a circular arrangement.  The individual SiPMs are single-cell SiPMs (Hamamatsu MPPC, S10931-50P) with a $3 \times 3\ mm^{2}$ active area.  Four individual SiPMs, grouped together in a linear array, are arranged such that they are parallel to the end of the upstream end of a scintillator as seen in figure \ref{fig:Labeled ST1}.
\begin{figure}[!htb]
	\centering
	\includegraphics[width=1.0\columnwidth]{design/figs/Labeled_ST1_Penny}
	\caption[ST1 of SiPM]{ST1 of SiPM.}
	\label{fig:Labeled ST1}
\end{figure}  

Four SiPMs, reading out one individual paddle, are current summed prior to pre-amplification.  The output of each preamp is then split, buffered for the Analog to Digital Converter (ADC) output, and amplified for the Time to Digital Converter (TDC) output by a factor five relative to the ADC.  The ADC outputs are readout \emph{via} flash ADCs (JLab 250 MHz Flash, fADC250), while the TDC outputs are input into leading edge discriminators (JLab LE discriminator), followed by a  32 channel flash TDC (TDC JLab F1-TDC).  Furthermore, each group of four SiPMs utilizes a thermocouple for temperature monitoring. There are 120 SiPMs in total, for a total of 30 pre-amplifier channels as seen figure \ref{fig:Start Counter Electronics}.
\begin{figure}[!htb]
	\centering
	\includegraphics[width=1.0\columnwidth]{design/figs/ST_Electronics}
	\caption[Start counter electronics]{Start counter electronics.}
	\label{fig:Start Counter Electronics}
\end{figure}

There are three components that comprise the SiPM detector and readout system.  The first component is the ST1 which holds 3 groups of 4 SiPMs.  The SiPMs are housed in a ceramic case, while being rigidly attached to the ST1.  In order to mimic the geometry of the 30 paddle design one group (of four SiPM's) is offset by $12^{\circ}$ relative to the central group, while another group is offset by $-12^{\circ}$.  One ST1 unit will collect light from three paddles individually.  The ST1 implements the current sum and bias distribution per group of 4 SiPMs.  It also has a thermocouple for temperature monitoring.  

The second component is the ST2 which is a Printed Circuit Board (PCB) that houses the electronics of the readout system.  It has 3 channels of pre-amplifiers, 3 buffers, and 3 factor five 
amplifiers.  Furthermore, it has 3 bias distribution channels with individual temperature compensation \emph{via} thermistors.  The ST2 is attached to the ST1 \emph{via} $90^{\circ}$ hermaphroditic connector.  

The third component, the ST3, provides interface to the power and bias supplies.  It also routes the 3 ADC, and 3 TDC outputs as well as the thermocouple output.  The ST3 connects to the ST2 emph{via} a signal cable assembly.  The ST3 is installed in the upstream chassis upstream of the Start Counter and next to the beam pipe seen in figure \ref{fig:SiPM Readout System}.
\begin{figure}[!htb]
	\centering
	\includegraphics[width=1.0\columnwidth]{design/figs/SiPM_Readout_System}
	\caption[SiPM Readout System]{SiPM Readout System.}
	\label{fig:SiPM Readout System}
\end{figure}