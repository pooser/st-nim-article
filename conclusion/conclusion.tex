\section{Conclusion} \label{sec:conclusion}

The \gx{} Start Counter was designed and constructed at Florida International University for use in Hall-D at TJNAF to provide separation of the 500 MHz photon beam bunch structure delivered by the CEBAF to within 99\% accuracy.  It is the first ``start counter'' detector to have utilized magnetic field insensitive SiPMs as the readout system.  Despite the many design and manufacturing complications, the ST has proven to have performed well beyond the design timing resolution of 350~ps with an average measured resolution of 250~ps.  Furthermore, the capabilities of the ST make it a viable candidate to assist in PID calculations.

The unique geometry of the ST nose section has illustrated the advantage of tapering trapezoidal geometry in thin scintillators.  Through simulation, tests on the bench, and analysis of data obtained with beam it has been definitively demonstrated that this geometry results in a phenomenon in which the amount of light detected increases as the scintillation source moves further downstream from the readout detector.

Since it's installation in Hall-D during the Fall 2014 commissioning run, the ST has shown no measurable signs of deterioration in performance.  This suggests that the ST scintillators are void of crazing and will most likely be able to meet and exceed the design performance well beyond the scheduled run periods associated with the \gx{} experiment.

It is planned to incorporate the ST into the level 1 trigger of the \gx{} experiment for high luminosity $(> 0.5\ \mu A)$ running.  Preliminary studies suggest that the high efficiency $(> 95\%)$ of the ST, in combination with the calorimeters, provides good suppression of electromagnetic background in regards to the level 1 trigger.  Furthermore, the ST's high degree of segmentation has shown to suppress various background contributions associated with complex topologies while simultaneously providing precision timing information for reconstructed charged particles in \gx{}.