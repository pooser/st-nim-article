\begin{abstract}
The design, fabrication, calibration, and performance of the GlueX Start Counter detector is described.  The GlueX experiment, staged in Hall D at the Thomas Jefferson National Accelerator Facility (TJNAF), primarily aims to study the rich spectrum of photo-produced mesons with unprecedented statistics.  The coherent bremsstrahlung technique is implemented in order to produce a 9 GeV linearly polarized photon beam incident on a liquid $\mathrm{H_{2}}$ target. A Start Counter detector was fabricated to properly identify the accelerator electron beam buckets and to provide accurate timing information. The Start Counter detector was designed to operate at photon intensities of up to $\mathrm{10^{8}\gamma/s}$ in the coherent peak and provides a timing resolution $\mathrm{<\ 300\ ps}$ so as to provide successful identification of the electron beam buckets to within 99\% accuracy. Furthermore, the Start Counter detector provides excellent solid angle coverage, $\sim 90 \%\ \mathrm{of}\ 4 \pi\ \mathrm{hermeticity}$, a high degree of segmentation for background rejection, and is utilized in the level 1 trigger for the experiment.  It consists of a cylindrical array of 30 thin scintillators with pointed ends that bend towards the beam at the downstream end. Magnetic field insensitive silicon photomultiplier detectors were selected as the readout system.\cite{Feynman1963118,Dirac1953888}
\end{abstract}